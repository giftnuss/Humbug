% Paket schulschriften
% Beispielfile zur  LaTeX-Anpassung 
%    Auszeichnung durch Unterstreichen
%    Umschaltung Schreibschrift/Normalschrift
% Version 1
% Walter Entenmann
% 17.10.2012
%
\documentclass[12pt]{article}
\usepackage[T1]{fontenc}
\usepackage[german]{babel}
\usepackage{schulschriften_ltx}        % Umschaltung 
                        % \normalschrift,\schreibschrift{<fontbefehl>}
                        % <fontbefehl>=\usefont{T1}{<font>}{<series>}{<shape>} 
                        % \thisfont (enthaelt den jeweils aktuellen Font) 
                        % Unterstreichen mit \underline{<text>}


\voffset-1in
\hoffset-1in
\textwidth16cm
\textheight25cm
\oddsidemargin2.5cm
\parindent1em
\sloppy

\newtheorem{satz}{Satz}

\begin{document}

% Schriftgroesse:
\fontsize{14pt}{18pt}\selectfont\par\parindent1em

%%%%%%%%%%% Umschaltung auf Schreibschrift und Schriftauswahl:
\schreibschrift{\usefont{T1}{wela}{m}{sl}}
%
\section{\underline{\strut Lesebuch in Schreibschrift}}
\subsection{\underline{\strut Rationalit\"at im Wandel der Zeit}}
 Aber worin besteht diese wohl schon so oft
diskutierte Rationalit\"at der Vorsokratiker?
\begin{enumerate}
\item Nach der Pause, wenn's geht, wird die Besprechung 
mit den G\"asten im Seminarraum stattfinden
\end{enumerate}
\begin{description}
\item\underline{\strut Vorwort} Das Buch beginnt mit einem Vorwort, 
in welchem die Zielsetzung und der wesentliche Inhalt
\end{description}

Und jetzt schalten wir wieder auf Normalschrift um.

%%%%%%%%%%% Umschaltung auf Normalschrift (Standard-LaTeX):
\normalschrift 
\section{Normalfont}
Und wieder ein Text in Standard-\LaTeX.
\end{document}
